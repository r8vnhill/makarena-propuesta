\subsubsection{Función de evaluación}
  \begin{definition}[\textbf{Recurso}\footnotemark]
    Se le dirá recurso a cada elemento en el juego que acerque a un jugador a una jugada óptima.
    Formalmente, diremos que un \textit{recurso} es un cambio en el estado del juego que incrementa
    las probabilidades de que el jugador que lo obtiene gane la partida.

    \textbf{Ejemplo}: En el ajedrez a cada pieza se le otorga un valor, siendo las \enquote{mejores 
    piezas} las que tienen un valor más alto.
    La pieza con valor más bajo es el \textit{peón} con un valor de 1; así podemos definir el valor
    de cada pieza en base al valor de un peón (diciendo que una pieza \(\mathbf{X}\) tiene un valor
    equivalente a \(x\) \textit{peones}).
    En este caso diremos que los recursos son los peones.
  \end{definition}
  \footnotetext{Término propio.}

  Se le dirá \textbf{función de evaluación}\cite{EvaluationFunction2022} \(v_i(s)\) a la función 
  que determina la cantidad de recursos que tiene el jugador \(i\) en el estado \(s\).
  A su vez, el \textbf{estado del juego} se define como el par \(a_i, a_{-i}\) que representa las
  acciones de los jugadores que llevan a una configuración específica del juego (un ejemplo de
  configuración puede ser la posición y cantidad de piezas en el tablero).