\subsubsection{Juegos de suma cero}
  Los \textit{\textbf{juegos de suma cero}} \cite{noauthor_zero-sum_2022} (\textit{zero sum games}) 
  son una representación matemática usada en \textit{teoría de juegos} y \textit{teoría 
  económica} para describir una situación donde existen dos \enquote{jugadores} que se enfrentan
  entre sí.
  Un \textit{juego es de suma cero} si la ventaja que lleva un jugador es igual a la pérdida del
  otro.

  Los \textit{juegos de suma cero} son un tipo de juegos de suma constante donde la suma de 
  ganancia y pérdida es igual a cero.

  En un juego de suma cero, todos los jugadores perciben la misma ganancia de cada recurso.   
  Para entender esto podemos considerar un juego con tres estados: empate, ganador y perdedor, 
  con puntajes 0, 1 y -1 respectivamente.

  \begin{itemize}
    \item \textbf{Empate:} Ningún jugador tiene ventaja; ambos jugadores tienen 0 puntos.
    \item \textbf{Ganador:} El jugador ganó, por lo que tiene 1 punto; el otro jugador pierde 
      así que tiene -1 puntos.
    \item \textbf{Perdedor:} El jugador perdió, por lo que tiene -1 puntos; el otro jugador ganó 
      así que tiene 1 punto.
  \end{itemize}
  
  De esta forma, para todo estado del juego, la suma de los puntajes es 0.