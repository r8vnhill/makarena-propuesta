\section{Metodología}
  El desarrollo del trabajo se guía fuertemente en los objetivos específicos de la tesis, de esta 
  forma se propone la división del trabajo en tareas atómicas, no necesariamente secuenciales:

  \begin{enumerate}
    \item Implementar el juego \enquote{Gato} que pueda ser jugado por dos jugadores (humanos o 
      artificiales).
    \item Implementar el juego \enquote{Ajedrez} que pueda ser jugado por dos jugadores (humanos o 
      artificiales).
    \item Implementar el algoritmo \textit{Minimax} para el juego \enquote{Gato} y el juego 
      \enquote{Ajedrez} con funciones de evaluación conocidas.
    \item Generalizar el algoritmo \textit{Minimax} para otros juegos (aquí no se busca implementar
      juegos adicionales a los ya mencionados).
    \item Implementar un algoritmo de GP simple (basado en el propuesto por Koza 
      \cite{kozaGeneticProgrammingMeans1994a}) que pueda encontrar una función de evaluación para
      los juegos mencionados.
    \item Hacer una revisión bibliográfica sobre técnicas de generalización de algoritmos de GP para 
      encontrar la(s) más adecuada(s) para el problema planteado.
    \item Reimplementar la solución original con las técnicas de generalización investigadas.
    \item Evaluar los resultados obtenidos por las funciones de evaluación conocidas y las obtenidas 
      mediante los algoritmos de GP mediante simulaciones de partidas entre jugadores artificiales
      para medir su eficacia.
    \item Comparar las funciones de evaluación obtenidas mediante GP con las conocidas en cuanto a
      similitud y complejidad.
  \end{enumerate}

  Todo este trabajo sería realizado con el lenguaje de programación \textit{Kotlin} 
  \cite{KotlinProgrammingLanguage} y el framework de algoritmos genéticos \textit{Jenetics} 
  \cite{wilhelmstotterJeneticsJavaGenetic}.

  Adicionalmente se plantea la posibilidad de hacer una revisión bibliográfica sobre usos de otros 
  tipos de \textit{Machine Learning} en videojuegos para tener una visión más completa sobre el 
  estado del arte.