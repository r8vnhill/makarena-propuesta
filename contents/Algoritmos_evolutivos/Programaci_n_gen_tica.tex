\subsubsection{Programación genética}
  La \textit{programación genética} (GP)\autocite{langdonFoundationsGeneticProgramming2013} es un tipo de algoritmo 
  genético donde el espacio de solución son \textbf{programas}.
  Para lograr esto se define a cada individuo como una representación abstracta de un programa, esta
  representación podría ser un \textbf{árbol de sintaxis abstracta}, una \textbf{pila de ejecución},
  entre otras.

  Un caso de uso típico de la \textit{programación genética} es hacer una \textit{regresión 
  simbólica}\autocite{poliFieldGuideGenetic2008} a partir de un conjunto de valores, donde el objetivo final es 
  encontrar una función que se aproxime bien a los valores de entrada.
  Esto será relevante para este trabajo, ya que el problema que se quiere resolver es encontrar una
  función de evaluación para el \textit{algoritmo Minimax}, esto se explicará en más detalle en la
  \cref*{sec:estado-del-arte}.
  
  % \ab{No esta claro como vas a usar algoritmo evolutivos, y con cual finalidad.}