\subsubsection{Algoritmos genéticos}
  Los \textit{algoritmos genéticos}\cite{holland_adaptation_1992, ahvanooey_survey_2019} son una 
  clase de algoritmos evolutivos que utilizan una \textit{población de individuos} para resolver un 
  problema.
  En este contexto, los individuos son una secuencia de valores que representan una 
  \textbf{solución al problema}.
  A cada individuo se le asigna una \textbf{aptitud} (\textit{fitness}) que es un valor numérico que 
  representa la calidad de la solución, de esta forma podremos discriminar qué soluciones son más
  apta para resolver el problema.

  Específicamente, los algoritmos genéticos pueden definirse de la siguiente forma:
  \begin{enumerate}
    \item Se crea una población inicial de soluciones aleatorias.
    \item Se seleccionan pares de individuos siguiendo alguna \textit{estrategia de 
      selección}.
    \item Se cruzan los pares de individuos siguiendo algún \textit{operador de cruza} para 
      obtener nuevos individuos (hijos).
    \item Se mutan los nuevos individuos siguiendo algún \textit{operador de mutación} y una 
      \textit{tasa de mutación}.\footnote{Probabilidad (usualmente baja) de que un individuo mute.}
    \item Se evalúa la solución más apta de la nueva población respecto a un criterio de aceptación
      (e.g., margen de error).
      Si se cumple el criterio, se termina el algoritmo, si no, se vuelve al paso 2.
  \end{enumerate}
  