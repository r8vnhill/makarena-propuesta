\section{Estado del arte}
  \label[section]{sec:estado-del-arte}
  \subsection{\textit{Artificial Ant Problem}}
    El \textbf{problema de las hormigas artificiales} 
    \autocite{jeffersonGenesysSystemEvolution1990,kozaGeneticProgrammingMeans1994a} es un problema 
    que busca encontrar un programa que, al ser 
    ejecutado para una hormiga, les permite encontrar toda la comida en una grilla con una cantidad 
    máxima de movimientos, considerando que la comida esté distribuida de acuerdo a un patrón.

    En este caso, se espera que el programa sea capaz de reconocer los patrones de comida para 
    encontrarla, podemos interpretar que este problema es un juego de estrategia de un jugador.
    Este problema puede ser reinterpretado como un juego de dos jugadores donde hay un adversario
    que no puede realizar movimientos y cada pieza de comida es un recurso; dicho de otra forma:
    el adversario tiene tantos recursos como la cantidad de comida que la hormiga aún no encuentra.
    De esta forma, el problema de la hormiga se reduce a un \textit{juego de suma zero} de dos 
    jugadores, por lo que encontrar una solución para este problema es equivalente a deducir una
    función de evaluación para \textit{Minimax}.
    
    En \cite{kozaGeneticProgrammingMeans1994a} \textit{Koza} propone una solución al problema de las
    hormigas artificiales que utiliza \textit{programación genética} para generar un programa con 
    las primitivas:
    \begin{itemize}
      \item \textit{Moverse hacia el frente}.
      \item \textit{Girar a la derecha}.
      \item \textit{Girar a la izquierda}.
      \item \textit{Comprobar} si tiene comida en la casilla del frente.
      \item \textit{Programas o funciones} de 2 ó 3 parámetros.
    \end{itemize}
 
    Sin embargo, si bien esta solución presentó resultados positivos, un análisis más exhaustivo
    mostró que la solución no es óptima \autocite{kuscuEvolvingGeneralisedBehaviour1998}.
    Estos estudios mostraron que la solución propuesta presenta un \textit{sesgo evolutivo} que
    evita la generalización de ésta a otros escenarios, principalmente porque no es capaz de
    encontrar una solución óptima para grillas que difieran de los datos con que se entrenó a la
    hormiga.
    Más aún, este problema de generalización \textbf{es común} en GP.

    Las siguientes secciones presentan soluciones a este problema de generalización para casos 
    generales y se plantearán como alternativas a la solución original de \textit{Koza}.

\ab{No es claro porque artificial ant problem es un trabajo relacionado...}

  \subsection{Regresión simbólica}
    \label{sec:regresion-simb}
    Dado que lo que buscamos es encontrar una función de evaluación, podemos reducir, tanto el 
    problema de la hormiga como la deducción de la función de evaluación, a un problema de regresión
    simbólica.
    El tópico de \textit{Programación Genética para Regresión Simbólica} (GPSR) se ha desarrollado 
    ampliamente para buscar soluciones que puedan ser generalizadas para resolver el problema que se
    planteó en la sección anterior.

    Un problema común que limita la capacidad de generalización de la regresión simbólica es que
    el modelo de GP tiende a aprender modelos demasiado complejos, estos modelos generan 
    soluciones específicas al dominio del set de datos con el que se entrenó (impidiendo la 
    generalización de la solución) \cite{raymondMultiobjectiveGeneticProgramming2021}.

    En este tópico se han propuesto múltiples soluciones, Raymond et. al. 
    \cite{raymondAdaptiveWeightedSplines2020,raymondMultiobjectiveGeneticProgramming2021} proponen
    una solución de interés para este trabajo ya que presenta una generalización con resultados
    favorables además de evitar que el modelo de GP aprenda modelos demasiado complejos.
    Para esto los autores utilizan una representación alternativa a la representación clásica de
    GP utilizando \textit{Splines} en lugar de árboles para representar el modelo de la solución.
    
  \subsection{\textit{Deep Learning} en videojuegos}
    
\ab{Quizas puedes no hablar de deep learning. Lleva mucho riesgos a hablar de esto. Quizas puedes hablar de reinforcement learning en vez (entiendo que minimax es un tipo de reinforcement learning}