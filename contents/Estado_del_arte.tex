\section{Estado del arte}
  \subsection{\textit{Artificial Ant Problem}}
    El \textbf{problema de las hormigas artificiales} \autocite{kozaGeneticProgrammingMeans1994} es 
    un problema que busca encontrar un programa que, al ser ejecutado para una hormiga le permita 
    encontrar toda la comida en un entorno dado, con una cantidad limitada de movimientos.
    En \autocite{kozaGeneticProgrammingMeans1994}, \textit{Koza} propone una solución al problema
    utilizando programación genética para encontrar un programa que maneje a la hormiga.
    Para esto, se definen las primitivas \cite{deapprojectArtificialAntProblem}:
    
    \begin{itemize}
      \item \mintinline{kotlin}{if (foodAhead) { child1() } else { child2() }}
      \item \mintinline{kotlin}{moveForward()}
      \item \mintinline{kotlin}{turnLeft()}
      \item \mintinline{kotlin}{turnRight()}
    \end{itemize}

    De esta forma, los programas que se generan mediante GP podrá moverse en 3 direcciones y 
    verificar si tiene comida en frente.

    La \textit{función de evaluación} simplemente contará la cantidad de comida acumulada por la 
    hormiga luego de una cierta cantidad de movimientos.
    De esta forma, el \textit{fitness} del programa será directamente proporcional a dicha cantidad
    de comida.
    Una implementación completa se puede encontrar en \cite{fortinDeapAntPy2022}.
  
  \subsection{Potencial de adaptabilidad}
    