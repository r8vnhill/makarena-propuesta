\section{Objetivos}
  \subsection{Objetivos generales}
    La investigación propuesta tiene como objetivo principal \textit{diseñar una técnica} 
    (algoritmo) que les permita a lxs desarrolladorxs de videojuegos, además de gente en las 
    áreas de investigación de videojuegos y teoría de juegos, definir una función de evaluación que 
    permita \textbf{desarrollar jugadores artificiales} utilizando \textit{Minimax} como algoritmo de decisión.
    
  \subsection{Objetivos específicos}
    Los objetivos específicos son los siguientes:
    
    \begin{itemize}
      \item Crear una \textbf{representación genérica} del \textit{estado de un juego} de suma cero
            de 2 jugadores en el que los jugadores toman turnos intercalados (e. g. si el jugador 
            \(J_a\) realiza una jugada, la siguiente jugada la hará su oponente \(J_b\)). 
            % \ab{Muy generico. Cual tipo de juego quieres enfocar?}
      \item Implementar el algoritmo \textit{Minimax} que utilice la representación de estados ya 
        mencionada.
      \item Definir e implementar la \textbf{deducción de funciones de evaluación} utilizando 
        \textit{Programación Genética}.
      \item Utilizar las funciones de evaluación deducidas en dos juegos distintos: \textit{Gato} 
        (\textit{Tic-Tac-Toe}) y \textit{Ajedrez}.
    \end{itemize}