\section{Resultados esperados}
  Para evaluar los resultados de este trabajo se utilizarán las implementaciones de los juegos 
  mencionados en la sección anterior para simular partidas entre jugadores artificiales.
  Para esto se utilizarán 3 tipos de jugadores artificiales: un jugador que tome decisiones 
  aleatorias, un jugador que tome decisiones basadas en el \textit{algoritmo Minimax} con una 
  función de evaluación tomada de la literatura (conocida), y uno que tome decisiones basadas en el 
  \textit{algoritmo Minimax} con funciones de evaluación obtenidas con el algoritmo de GP.
  
  Los jugadores artificiales luego pueden evaluarse en base a su \textit{ratio de victorias} 
  (\textit{win rate}) enfrentandose a otros jugadores artificiales (tanto del mismo tipo 
  como de los otros).
  En este caso, se espera que los jugadores basados en \textit{Minimax} tengan un \textit{win rate}
  superior a los que toman decisiones aleatorias, y que los jugadores con funciones de evaluación
  deducidas con GP tengan ratios de victoria similares a los que utilizan funciones de evaluación
  conocidas.

  Además, se compararán las funciones \textit{deducidas} con las \textit{conocidas} en cuanto a su
  similitud y complejidad.
  En este caso, se espera que las funciones de evaluación deducidas tengan una complejidad similar a
  las de las funciones de evaluación conocidas.