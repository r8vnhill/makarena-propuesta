\section{Introducción}
  El concepto de \textit{máquinas que aprenden} fue propuesta por primera vez \textit{Alan Turing} 
  en 1950 
  con una simple pregunta: \enquote{¿Pueden las máquinas pensar?} \cite{turing_icomputing_1950}.
  Siguiendo esa pregunta, no es descabellado preguntarse: \enquote{¿Puede un computador jugar como 
  una persona?}

  El estudio de \textit{teoría de juegos} se le atribuye a \textit{von Neumann} en su paper de 
  1928 \cite{v_neumann_zur_1928} y desde entonces se ha desarrollado como un área de investigación.
  Entre los avances de la teoría de juegos se encuentra el desarrollo de algoritmos capaces de 
  idear estrategias similares a las de los jugadores humanos, llamaremos a estos 
  \enquote{\textit{jugadores artificiales}}.

  En este documento nos centraremos en un algoritmo en particular, \textit{Minimax}, debido a que ha
  sido ampliamente estudiado y utilizado \cite{v_neumann_zur_1928,fan_minimax_1953,
  maschler_game_2013,thekumparampil_efficient_2019,noauthor_minimax_2022}.
  Sin embargo, este algoritmo depende de una \textit{función de evaluación} que otorga un valor
  numérico a cada estado del juego para discriminar qué jugada es \enquote{mejor} que otra, pero 
  esta función se desprende de estudiar en detalle el juego \cite{shannon_programming_1988} o por 
  \enquote{tanteo} dado que la función de evaluación es \textbf{específica del dominio} 
  \cite{charles_r_dyer_cs_nodate}.
  Este trabajo propone una manera automática de encontrar la función de evaluación del estado del
  juego sin importar el dominio de éste.