\section{Introducción}
  El concepto de \textit{máquinas que aprenden} fue propuesto por primera vez por \textit{Alan 
  Turing} en 1950 con una simple pregunta: \enquote{¿Pueden las máquinas pensar?} 
  \autocite{turingCOMPUTINGMACHINERYINTELLIGENCE1950}.
  Siguiendo esa pregunta, no es descabellado preguntarse: \enquote{¿Puede un computador jugar como 
  una persona?}

  El inicio del estudio de \textit{teoría de juegos} se le atribuye a \textit{von Neumann} en su 
  publicación de 1928 \autocite{v.neumannZurTheorieGesellschaftsspiele1928} y desde entonces se ha 
  desarrollado como un área de investigación.
  Entre los avances de la teoría de juegos se encuentra el desarrollo de algoritmos capaces de 
  idear estrategias similares a las de los jugadores humanos, llamaremos a estos 
  \enquote{\textit{jugadores artificiales}}.

  En este documento nos centraremos en un algoritmo en particular, \textit{Minimax}, debido a que ha
  sido ampliamente estudiado y utilizado \autocite{v.neumannZurTheorieGesellschaftsspiele1928,
  fanMinimaxTheorems1953,maschlerGameTheory2013,thekumparampilEfficientAlgorithmsSmooth2019,
  Minimax2022}.
  Sin embargo, este algoritmo depende de una \textit{función de evaluación} que otorga un valor
  numérico a cada estado del juego para discriminar qué jugada es \enquote{mejor} que otra, pero 
  esta función se desprende de estudiar en detalle el juego 
  \autocite{shannonProgrammingComputerPlaying1988} o por \enquote{tanteo} dado que la función de 
  evaluación es \textbf{específica del dominio}\autocite{dyerCS540Lecture}.
  Este trabajo propone una \textit{\textbf{manera automática} de encontrar la función de evaluación 
  del estado del juego sin importar el dominio de éste}.