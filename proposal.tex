%% "Makarena" (c) by Ignacio Slater M.
%% "Makarena" is licensed under a
%% Creative Commons Attribution 4.0 International License.
%% You should have received a copy of the license along with this
%% work. If not, see <https://creativecommons.org/licenses/by/4.0/>.
\documentclass{article}
  \usepackage{import} % This will make our life easier
  \import{.preamble}{Packages}
  \import{.preamble}{Definitions}
  \import{.preamble}{config}

  \setup { 
    title = {
      \textit{Makarena}: Deducción evolutiva de funciones de evaluación aplicadas a \textit{Minimax}
    },
    subtitle = {Propuesta de tesis},
    author = {
      \textbf{Ignacio Slater M.} \\
      \textit{Departamento de Ciencias de la Computación} \\
      Universidad de Chile
    },
    advisors = {
      \textbf{Nancy Hitschfeld} \\
      \textit{Departamento de Ciencias de la Computación} \\
      Universidad de Chile  \\
      {\small\textit{Profesora guía}}
    },
    logo = {logos/LogoDCC.pdf},
    location = {Santiago, Chile},
    date = \today
  }

\begin{document}
  \begin{titlepage}
    \centering
    % Some vertical space before the title (note the *, it's needed to add space at the beginning 
    % of the page).
    \vspace*{2cm}
    \titleblock [2cm]
    \inputlogo
    \vspace{1cm}  % Some vertical space after the logo.
    \authorblock
    \vfill  % Fills the page so the location and date are at the bottom.
    \location \\
    \dateblock  \\
    \footnotesize\texttt{Versión 0.1.2204231628}
  \end{titlepage}

 
  \import{contents}{Introducci_n}
  \import{contents}{Planteamiento_del_problema}

  \section{Estado del arte}
    \Blinddescription

  \section{Pregunta de investigación}
    Dada una representación del estado de un juego. 
    ¿Es posible definir un algoritmo que encuentre una función de evaluación del estado del juego 
    para la toma de decisiones del algoritmo \textit{Minimax}?

  \section{Hipótesis}
    Es posible utilizar \textit{programación genética} para derivar la función de evaluación como
    un árbol de sintaxis abstracta.

  \section{Objetivos}
    \blinditemize

  \section{Metodologías}
    \Blindtext

  \section{Resultados esperados}
    \blindtext

  % \nocite{*}
  \printbibliography
\end{document}
